\begin{abstract}

\addchaptertocentry{\abstractname}

In order for researchers to study the code of open source projects, they must first select a sample of repositories to study.
While there are many tools that can aid in this selection, they tend share several flaws.
They are generally not user-friendly, impose limitations to the allowed sample size, or do not allow the user to filter out data points that are less than relevant to their needs.

The goal of this project is to produce a full-stack web application that would address these limitations, allowing the end-user to easily select existing open-source projects from the most popular online project ecosystem: GitHub.
The application back-end would continuously mine public repositories through the use of GitHub's API, and store their statistical information to a MySQL server database.
Through the use of a search engine interface, the user would be able to extract a repository sample that fits their specific requirements.
These search results would both be immediately presentable in the browser, and exportable in various file formats fit for either direct analysis, or further use in other applications.

\end{abstract}