\chapter{Project Requirements and Timeline}

\section{Goals}

Having outlined the issues of existing tools, it became apparent that in order to address them, our finished product needs to:

\begin{itemize}
    \item Use a combination of GitHub's API and front-end mining to acquire all the publicly available information on the site;
    \item Present all the mined repository information in a clear and concise manner;
    \item Offer an intuitive and easy to use web interface (UI) for querying;
    \item Provide the user with immediate access to the full set of the search results;
    \item Allow the user to export the entire search result set.
\end{itemize}

While most of these demands are somewhat easy to meet, their success heavily depends on how the mining algorithm and application back-end (database schema and API) are implemented.
The exact implementation details will be discussed in the upcoming chapter.

\section{Tools and Technologies Used}

It was decided at during the early stages of the project that the back-end was to be written in Java, using the popular Spring Boot Framework.
Seeing as though it is one of the most commonly used and well-documented frameworks for back-end web development, using it only made sense.
Additional plugins necessary for the implementation consisted of:
\begin{itemize}
    \item Java Persistence API (JPA)\footnote{\url{https://www.oracle.com/java/technologies/persistence-jsp.html}} for managing the relational schema;
    \item Hibernate\footnote{\url{https://hibernate.org}} as the object-relational mapping (ORM) tool;
    \item Flyway\footnote{\url{https://flywaydb.org/documentation/plugins/springboot}} for performing database migrations;
    \item Jsoup\footnote{\url{https://jsoup.org}} and Selenium\footnote{\url{https://www.selenium.dev}} for web-page mining;
    \item OkHttp3\footnote{\url{https://square.github.io/okhttp}} for making HyperText Transfer Protocol (HTTP) requests to the GitHub API;
    \item Jackson\footnote{\url{https://github.com/FasterXML/jackson}} and OpenCSV\footnote{\url{http://opencsv.sourceforge.net}} for converting the data into text-based formats;
    \item Lombok\footnote{\url{https://projectlombok.org}} to ease development by reducing the amount of ``boilerplate code'' \footnote{\url{https://en.wikipedia.org/wiki/Boilerplate_code}}.
\end{itemize}

While I was given more freedom to choose any framework for developing the front-end, my supervisor advised me to ``keep it simple'' and use the popular Bootstrap framework.
Although I had limited knowledge of said framework, using it significantly eased development due to its overall accessibility.
While it did offer a plethora of functionalities out of the box, several additional widgets were used to enhance the user experience:
\begin{itemize}
    \item Bootstrap-datepicker\footnote{\url{https://github.com/uxsolutions/bootstrap-datepicker}} for easier date selection in date inputs;
    \item Chart.js\footnote{\url{https://www.chartjs.org}} to better display statistical data;
    \item Typeahead.js\footnote{\url{https://github.com/bassjobsen/Bootstrap-3-Typeahead}} for input suggestions;
    \item Font Awesome\footnote{\url{https://fontawesome.com}} and GitHub Octicons\footnote{\url{https://primer.style/octicons}} for more icon options.
\end{itemize}

Although the initial suggestion was to deploy the application in a single Docker container, it was easier to write a compose configuration and deploy each of its major components (front-end, back-end and database) in their own respective containers. The three containers include:
\begin{itemize}
    \item A database container, which based on the \textit{mysql}\footnote{\url{https://hub.docker.com/_/mysql}} image hosts the application database;
    \item An application container, which based on a \textit{maven-chrome}\footnote{\url{markhobson/maven-chrome
    }} image hosts the application back-end;
    \item A front-end container, which based on an \textit{nginx}\footnote{\url{https://hub.docker.com/_/nginx}} image supplies the client with the web page HTML\@.
\end{itemize}

\section{Development Timeline}

The development of the project can be broken up into several month-log phases:
\begin{enumerate}
    \item Phase ``zero'', spanning from mid to late February, was the preparation phase. It was during this time that the project description and study materials were provided to me.
    \item Phase one, spanning the month of March and first week of April was dedicated to the development of the application back-end. Everything, from the crawling algorithm, to the database schema and application API was done in this period.
    \item Phase two, spanning the rest of April and the first week of May, was dedicated to the implementation of the front-end web interface.
    \item Phase three, spanning the remainder of the timeline, was dedicated to the deployment, and writing the report.
\end{enumerate}